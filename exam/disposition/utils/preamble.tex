\documentclass[
	10pt,
	a4paper,
]{article}

% Random text
\usepackage{blindtext}
\usepackage{lipsum}

% Making fancy header and footers
\usepackage{fancyhdr}

% Danish letters
\usepackage[utf8]{inputenc}

% Reset footnote counter on new page
\usepackage[bottom, perpage]{footmisc}

% Math symbols
\usepackage{amssymb}

% For coloring \ref{}
\usepackage{hyperref}
\hypersetup{
	colorlinks,
	linkcolor={red!50!black},
	citecolor={blue!50!black},
	urlcolor={blue!80!black}
}

% Dots in Content
\usepackage{tocloft}

% Enable \todo{<task>}
\usepackage{todonotes}

% Adds Bibliography section to Contents
\usepackage[nottoc,notlof,notlot]{tocbibind}

% For Glossaries and addes them to Contents
\usepackage[toc]{glossaries}

% Insert PDF as page
\usepackage{pdfpages}

% For pretty tables
\usepackage{booktabs}
\usepackage{tabu}
\setlength{\tabcolsep}{10pt}
\renewcommand{\arraystretch}{1.2}

% Page margins
\usepackage{geometry}
\geometry{a4paper, left	= 3cm, right = 3cm, top = 4cm,bottom = 4cm}

% Getting name of currect section with \currentname
\usepackage{nameref}
\makeatletter
\newcommand*{\currentname}{\@currentlabelname}
\makeatother

% Enable \today
\usepackage{datetime}

% Enables \url{}
\usepackage{url}

% For translating document tag titles like 'Indholdsfortegnelse'
% instead of 'Contents', used below
\usepackage[english]{babel}
\addto\captionsenglish{\renewcommand{\contentsname}{Indholdsfortegnelse}}
\addto\captionsenglish{\renewcommand{\chaptername}{Kapitel}}
\addto\captionsenglish{\renewcommand{\bibname}{Litteraturliste}}
%\addto\captionsenglish{\renewcommand{\mtctitle}{Kapitel indhold}}

% For removing "Chapter X" from chapter headings
\usepackage{titlesec}
\titleformat{\chapter}{\huge\bf}{\thechapter.}{20pt}{\huge\bf}

% Increase rowspacing
\addtolength{\jot}{1em}

% Unknow packages
\usepackage[pages=some]{background}
\usepackage{subcaption}
\usepackage{nopageno}
\usepackage{graphicx}
\usepackage{lastpage}
\usepackage{caption}
\usepackage{amsmath}
\usepackage{wrapfig}
\usepackage{float}
\usepackage{multicol}
