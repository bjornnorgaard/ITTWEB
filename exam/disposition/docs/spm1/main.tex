\section{ASP.NET Core}

\paragraph{Spørgsmål}
Redegøre for principperne og vigtigste karakteristika for webudvikling med brug af et server-side MVC framework med udgangspunkt i ASP.NET Core. Samt vis hvordan man kan designe og implementere en	webløsning, som omfatter persistering af data i en relationel database med anvendelse af Entity Framworket.

\subsection{Frameworket}
ASP.NET Core er en ny generation af ASP.NET, udviklet af Microsoft. Frameworket er open-source og cross-platform.
Der er udviklet et CLI, som kan bruges i stedet for Visual Studio, f.eks. til oprettelse af projekter, database scaffolding, byg osv.
I modsætning til ASP.NET er Core frameworket lightweight og leverer bedre perfomrance.

\subsection{MVC}
Er \textit{convention over configuration}. Er et arkitekturelt pattern som opdeler en applikation i tre forbundne dele.

\paragraph{Model} Repræsenterer domænet og bruges ved hjælp af EF til at persistere data.

\paragraph{View} Vises til brugeren. Laves med Razor views, hvor der kan laves for-løkke m.m.

\paragraph{Controller} Eksekvere forretningslogikken og sørger for at opdatere view på baggrund af ændringer i modellen og vice versa.

\begin{itemize}
	\item Typisk \textit{Controller/Action/Id}
\end{itemize}

\subsection{Anatomy}
Frameworket er cloud-ready med inbygget environment based configurationer. Der er også indbygget dependency injection.

ASP.NET Core apps består som minimum af to klasser:

\paragraph{Program.cs}
En ASP.NET Core app er en konsol app der laver en webserver i main metoden.

\paragraph{Startup.cs}
Her konfigureres dependency injection, middleware og services til forretningslogik.

\subsection{Entity Framework}
EF er en ORM, som gør det muligt at oprette en database på baggrund af C\# klasser.

\paragraph{DbContext}
Repræsenterer \textit{unit of work} og \textit{repository pattern}, sådan at den kan bruges til at kalde databasen.

\paragraph{Migration} Beskriver en ændring af modellen og hvordan den skal opdatere databasen.
